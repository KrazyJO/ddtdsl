\documentclass[a4paper,12pt,titlepage]{article}
\usepackage{ngerman}
\usepackage[utf8]{inputenc}
\usepackage{listings}
\pagestyle{headings}
\usepackage{graphicx}
\usepackage{lstcustom}

%Sprachdefinitionen
\lstloadlanguages{Java, XML}
\lstset{
	language=Java,
	showspaces=false,
	showstringspaces=false,
	basicstyle=\small\ttfamily\footnotesize,
	columns=fullflexible,% typewriter font look better with fullflex
	keywordstyle=\color[rgb]{0.627,0.165,0.467},
	stringstyle=\color[rgb]{0.286,0.172,0.980},
	commentstyle=\color[rgb]{0.294,0.565,0.478},
	tabsize=2,
	numbers=left,
	numberstyle=\scriptsize,
	breaklines=true,
	identifierstyle=\ttfamily,
	backgroundcolor=\color[rgb]{0.99, 0.99, 0.99},
	rulesepcolor=\color[rgb]{0.3, 0.3, 0.3},
	captionpos=b,
	frame=shadowbox,
	otherkeywords={@SuppressWarnings,@Singleton,@Check,@Inject}
}
%praeambel zu ende
\begin{document}
\tableofcontents
\newpage
\section{Einführung / Ziele}
\begin{lstlisting}[language=Java, caption={Erster Test}]
	class Testclass
	{
		String value = "Stringvalue";
	}
\end{lstlisting}
\section{DDTDSL}
\subsection{Objektbeschreibung}
\subsubsection{Node}
\subsubsection{Next}
\subsubsection{Attribute}
\subsubsection{Maybe}
\subsubsection{Many}
\subsection{Stringbeschreibung}
\subsubsection{Key}
\subsubsection{Value}
\subsubsection{Variablen}
\subsubsection{Maybe}
\subsubsection{Many}
\subsubsection{Or}
\section{Verwendete Technologien}
\subsection{Xtext}
\subsection{Xtend}
\subsection{Java Reflection}
\section{Programming - Compilergenerator}
\subsection{Was ist ein Compilergenerator?}
\subsection{Fehlerbehandlung}
\subsection{SimpleScanner}
\subsection{Objekte}
\subsubsection{Node}
\subsubsection{Next}
\subsubsection{Attribute}
\subsubsection{Maybe}
\subsubsection{Many}
\subsubsection{Parserkreise}
\subsection{Strings}
\subsubsection{Key}
\subsubsection{Value}
\subsubsection{Variablen}
\subsubsection{Maybe}
\subsubsection{Many}
\subsubsection{Or}
\section{Testing}
\subsection{Validation - Generator Syntax}
\subsection{JUnit - Generator Model}
\subsection{JUnit - Compiler}
\section{Anwedungsbeispiel}
\section{Fazit}
\section{Anhang}
\end{document}